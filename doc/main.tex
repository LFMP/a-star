\documentclass[12pt]{article}
\usepackage{sbc-template}
\usepackage{graphicx,float,url}
\usepackage[brazil]{babel}   
\usepackage[utf8]{inputenc}  
\usepackage{indentfirst}
\usepackage{graphicx}
\usepackage[table]{xcolor}


\title{Comparação de desempenho de diferentes heurísticas no algoritmos A*}

\author{Luiz Fellipe Machi Pereira\inst{1}, Mateus Tenório dos Santos\inst{1}}

\address{
Departamento de Informática -- Universidade Estadual de Maringá (UEM)\\ Maringá -- PR -- Brasil \email{ra103491@uem.br,ra99829@uem.br}}

\begin{document} 

\maketitle

\section{Introdução}

Descrito em 1968 por \textit{Peter Hart, Niss Nilsson} e \textit{Bertram Raphael} como \textit{Algoritmo A}\footnote{Algoritmo A*, WikiPedia \url{https://pt.wikipedia.org/wiki/Algoritmo_A*}}, o algoritmo A* (que é o algoritmo A utilizando de heurística apropriada para atingir um comportamento ótimo ) trata-se de um algoritmo para realizar a busca de caminho utilizando uma combinação de aproximações heurísticas do algoritmo de Busca em Largura e do Algoritmo de Dijkstra. Ele realiza a busca em um grafo começando de um vértice inicial com destino o vértice desejado. 

As aplicações do algoritmo A* vão desde encontrar rotas de deslocamento entre localidades a resolução de problemas, como a resolução de quebra-cabeças, que é onde o N-puzzle (ou no nosso caso, 15-puzzle) se encaixa. O único "problema" do algoritmo A*, é que sua árvore de busca gerada pode crescer indefinidamente até que o estado desejado seja encontrado, como descrito por \cite{Culberson94efficientlysearching}, dessa forma, aumentando o consumo de memória e tempo.

Um dos principais modificadores no desempenho do algoritmo A* é a heurística utilizada, onde diferentes heurísticas levam o algoritmo tomar diferentes caminhos. O objetivo desse trabalho é analisar o comportamento do algoritmo A* utilizando 5 heurísticas diferentes, onde poderemos analisar como cada heurística impacta no tempo e no uso de memória. Para isso, utilizaremos do algoritmo A* para resolvermos o quebra-cabeça 15-puzzle, aplicando as seguintes heurísticas : número de peças foras de seu lugar, número de peças fora de ordem na sequência numérica das 15 peças(seguindo a
ordem das posições no tabuleiro), somatório da Distância Manhattan para cada peça fora do lugar, combinação linear entre heurísticas e máximo valor entre heurísticas.

\section{15-puzzle}

O 15-puzzle, também conhecido como jogo do 15, é um quebra-cabeça baseado em deslizar peças em um tabuleiro que encontram-se fora de ordem. O objetivo do jogo é colocar as peças em ordem, fazendo movimento deslizantes que usam o espaço vazio. Um exemplo do tabuleiro pode ser visto na Tabela~\ref{tab:tabex} e na Tabela~\ref{tab:confinal}. Este jogo é um problema clássico para modelagem de algoritmos envolvendo heurísticas.

\begin{table}[H]
    \centering
    \begin{tabular}{|c|c|c|c|}
        \hline
        1 & 2 & 3 & 4 \\ \hline
        5 & 6 & 7 & 8 \\ \hline
        9 & 10 & 11 & 12 \\ \hline
        13 & 15 & 14 & 0 \\ \hline
    \end{tabular}
    \caption{Tabuleiro desordenado}
    \label{tab:tabex}
\end{table}

\begin{table}[H]
    \centering
    \begin{tabular}{|c|c|c|c|}
        \hline
        1 & 2 & 3 & 4 \\ \hline
        5 & 6 & 7 & 8 \\ \hline
        9 & 10 & 11 & 12 \\ \hline
        13 & 14 & 15 & 0 \\ \hline
    \end{tabular}
    \caption{Configuração final}
    \label{tab:confinal}
\end{table}

\section{Algoritmo A*}

O algoritmo A* é um algoritmo heurístico de busca que visa encontrar um caminho para o nó objetivo com o menor custo. O algoritmo A* consegue isso através da manutenção de uma árvore (que cresce em uma borda a cada iteração) de caminhos originados desde o nó inicial. Esta árvore cresce indefinidamente até que o nó objetivo seja alcançado.

Para determinar qual caminho o algoritmo deve percorrer, é realizado o cálculo de uma função $f(n)$ em cada iteração do algoritmo. Essa função guia o A* dando um valor estimado do custo do nó n até o nó objetivo. A função $f(n)$ é dada por : $$f(n) = g(n) + h(n)$$\\onde:
\begin{itemize}
    \item $n$ representa um estado/nó qualquer;
    \item $g(n)$ é o custo do nó inicial até o nó n dado;
    \item $h(n)$ é uma estimativa dada por uma função heurística do custo do nó $n$ até o nó objetivo.
\end{itemize}

\section{Metodologia}

O algoritmo A*, bem como as heurísticas foram implementadas utilizando a linguagem Python. Para aferir a quantidade de memória utilizada e o tempo necessário para computação da resposta utilizou-se a ferramente \textit{GNU time}. A máquina utilizada foi uma instância \textit{VM}, disponível na plataforma Google Cloud, com 1 vCPUj. 125 GB de memória RAM e sistema operacional Ubuntu 19.04.

\subsection{Estrutura de dados utilizada}

O algoritmo A* durante seu processamento gera muitos estados, cada estado gerado é armazenado um tabuleiro e as informações de $f(n)$, $g(n)$, $h(n)$ e lista de sucessores. São três os conjuntos para armazenamento de estados, ou nós, gerados pelo algoritmo A*, sendo eles os conjuntos $A$, $F$ e $\Gamma$.
\begin{itemize}
    \item Conjunto $A$ - Conjunto dos estados abertos: são os estados que já foram gerados, mas ainda não foram processados pelo algoritmo;
    \item Conjunto $F$ - Conjunto dos estados fechados: aqueles que já foram processados pelo algoritmo. Pode ser tratado também como o conjunto dos estados que formam o caminho do estado inicial até o estado objetivo;
    \item Conjunto $\Gamma$ - Conjunto dos estados sucessores: é o conjunto de estados sucessores de um determinado estado v.
\end{itemize}

Porém, como citado anteriormente, devemos tomar cuidado no momento de implementação, pois a árvore gerada pelo algoritmo A* cresce rapidamente, devido a grande quantidade de estados gerados por iteração. Portanto, é necessário utilizarmos de estruturas de dados que possuam bons tempos de acesso, inserção e remoção.

A estrutura escolhida para armazenar os conjuntos $A$ e $F$ foi a o dicionário, que é padrão da linguagem Python. Um dicionário permite armazenar um par chave-valor, passando a chave de acesso é possível realize várias operações com complexidade $\mathcal{O}(1)$ em relação ao tempo, sendo elas armazenar um item, recuperar um item pela chave, verificar se a chave está contida no dicionário e remover um item. Verificar se um objeto está entre os valores do dicionário possui complexidade $\mathcal{O}(n)$ em relação ao tempo. \footnote{Time Complexities of the Dictionary’s Functions: \url{https://www.pythonlikeyoumeanit.com/Module2_EssentialsOfPython/DataStructures_II_Dictionaries.html}} 

Contudo em cada iteração do algoritmo é necessário saber qual elemento possui o menor valor de $f(n)$. Para que não fosse necessário iterar sobre o conjunto $A$, operação feita em $\mathcal{O}(n)$,  usou-se a estrutura heap mínimo, ou \textit{heap queue}, que guarda uma cópia do conjunto mapeando o valor de $f(n)$ com o tabuleiro correspondente. Desta forma o acesso ao elemento com menor valor de $f(n)$ é constante, uma vez que este é o primeiro elemento da estrutura\footnote{Heap queue algorithm: \url{https://docs.python.org/2/library/heapq.html}}.

Finalmente, para armazenar o conjunto $\Gamma$ utilizou-se da estrutura lista, uma vez que se faz necessário iterar sobre o conjunto na execução do algoritmo, sendo inevitável um tempo de execução linear.

\subsection{Heurísticas}

Esta sessão será dedicada a apresentação das heurísticas utilizadas na implementação do algoritmo A*.

\subsubsection{$h_{1}'(n)$ - \textbf{Número de peças foras de seu lugar}}

A heurística $h_{1}'(n)$ consiste na contagem das peças que estão fora de seu lugar na configuração final (tabela \ref{tab:confinal}). No exemplo da tabela \ref{tab:tabex} a heurística retorna o valor 2 uma vez que as peças 14 e 15 estão fora de lugar. A complexidade é linear em relação ao tamanho do tabuleiro, uma vez que é necessário percorre-lo sequencialmente para encontrar quais peças estão fora do devido lugar. Neste caso o tempo de execução é constante devido a não variação do tamanho do tabuleiro.

\subsubsection{$h_{2}'(n)$ - \textbf{Número de peças fora de ordem na sequência numérica}}

Esta heurística consiste na contagem das peças que estão fora de seu lugar na sequência numérica das 15 peças. Vamos supor a configuração inicial da tabela \ref{tab:tab3}:

\begin{table}[H]
    \centering
    \begin{tabular}{|c|c|c|c|}
        \hline
        2 & 1 & 5 & 13 \\ \hline
        0 & 3 & 4 & 14 \\ \hline
        15 & 8 & 9 & 10 \\ \hline
        6 & 7 & 11 & 12 \\ \hline
    \end{tabular}
    \caption{Tabuleiro desordenado}
    \label{tab:tab3}
\end{table}

Primeiro passo considerar a sequências de “posições” no tabuleiro conforme a sequência “numérica” esperada para o estado final, ou seja, iniciando na posição [0,0] (linha 0 e coluna 0) e terminando na posição com o zero (vazio). Levando isso em consideração, obteremos a seguinte sequencia : 

\begin{table}[H]
    \centering
    \begin{tabular}{|c|c|c|c|c|c|c|c|c|c|c|c|c|c|c|c|}
        \hline
        2 & \cellcolor{yellow!25}1 & \cellcolor{yellow!25}5 & \cellcolor{yellow!25}13 & \cellcolor{yellow!25}0 & 3 & 4 & \cellcolor{yellow!25}14 & 15 & \cellcolor{yellow!25}8 & 9 & 10 & \cellcolor{yellow!25}6 & 7 & \cellcolor{yellow!25}11 & 12 \\ \hline
    \end{tabular}
    \caption{Sequência gerada a partir do tabuleiro desordenado}
    \label{tab4}
\end{table}

Agora, é analisada a sequência numérica. As células em destacadas são as posições que são contabilizadas para a heurística. Podemos ter duas ocorrências nesta heurística. A primeira é que a posição atual é inferior a posição anterior, e a segunda é que a posição atual não é a próxima posição em comparação com a anterior, ou seja, pos[atual] $>$ pos[anterior + 1]. É válido notar que após o 0, ou posição vazia, não é contabilizada uma ocorrência. Neste exemplo, a heurística deve retornar como valor 8.

A complexidade da Heurística é linear ao tamanho do tabuleiro. Como o tamanho do tabuleiro não varia, o tempo de execução é constante.

\subsubsection{$h_{3}'(n)$ - \textbf{Somatório da Distância Manhattan}}

Esta heurística consiste em realizar o cálculo da Distância Manhattan para cada peça encontrada fora de seu lugar, tendo como base o tabuleiro final esperado. Dessa forma, o valor da heurística é a soma de todas as Distâncias Manhattan aplicadas.

De maneira formal, definimos a distância de Manhattan entre dois pontos num espaço euclidiano com um sistema cartesiano de coordenadas fixo como a soma dos comprimentos da projeção da linha que une os pontos com os eixos das coordenadas. \footnote{Distância Manhattan e Geometria Pombalina, Wikipédia. \url{https://pt.wikipedia.org/wiki/Geometria_pombalina}}

Por exemplo, num plano que contem os pontos $P1$ e $P2$, respectivamente com as coordenadas $(x1, y1)$ e $(x2, y2)$, é definido por : $$|x1 - x2| + |y1 - y2|$$

No caso desse trabalho, temos como P1 as coordenadas da peça fora do lugar e P2 as coordenadas da respectiva peça no tabuleiro de configuração final. Essa Heurística é linear no número de peças do tabuleiro.

\subsubsection{$h_{4}'(n)$ - \textbf{Combinação linear entre heurísticas}}

Nesta heurística é realizada uma combinação linear entre as heurísticas anteriores, onde a soma dos coeficientes p1, p2 e p3 deve ser igual a 1. $$p1 * h1(n) + p2 * h2(n) + p3 * h3(n)$$ \\ Onde $p1 + p2 + p3$ = $1$

A escolha dos pesos é feita com base no seu comportamento em testes anteriores, e os pesos escolhidos foram $p1 = 0,4$, $p2 = 0,2$ e $p3 = 0,4$. A heurística é linear em relação ao tamanho do tabuleiro, porém, como é necessário computar as heurísticas anteriores 

\subsubsection{$h_{5}'(n)$ - \textbf{Valor máximo entre heurísticas}}

É definida como o valor máximo entre as heurísticas 1, 2 e 3. A complexidade em relação ao tempo mantém-se igual a das heurísticas utilizadas.

\subsection{Casos de teste e métricas}\label{subsec:testmetricas}

Para realizar a comparação de desempenho entre as diferentes heurísticas no algoritmo A* faz-se necessário a utilização de métricas para que possam ser confrontadas. Neste trabalho foram utilizadas as métricas consumo de tempo e consumo de memória.

O conjunto de caso de testes utilizados para execução e aplicação das métricas estão dispostos na Tabela~\ref{tab:testes}.

\begin{table}[H]
\centering
\begin{tabular}{cc}
\hline
\textbf{Teste} & \textbf{Configuração inicial} \\ \hline
1 & 6 5 1 2 9 7 4 3 14 13 10 11 0 15 12 8 \\ \hline
2 & 6 5 1 2 9 10 7 3 13 11 15  4 14 12 0 8 \\ \hline
3 & 3 4 8 0 2 7 12 11 6 5 10 15 1 9 13 14 \\ \hline
4 & 5 7 15 4 3 10 8 11 13 9 6 12 2 14 0 1 \\ \hline
5 & 5 3 0 12 13 1 8 7 14 9 4 6 15 10 11 2 \\ \hline
6 & 4 15 9 6 3 5 10 8 2 7 12 14 11 0 1 13 \\ \hline
7 & 10 1 6 2 3 7 4 8 9 5 14 12 13 11 15 0 \\ \hline
8 & 6 7 8 12 2 3 4 0 1 13 14 15 5 9 10 11 \\ \hline
9 & 11 0 4 12 1 14 8 10 3 6 7 15 2 5 9 13 \\ \hline
10 & 15 14 13 0 12 11 10 5 4 8 3 6 9 7 2 1 \\ \hline
\end{tabular}
\caption{Casos de testes utilizados}
\label{tab:testes}
\end{table}

Dado que a máquina utilizada para todos os casos de teste e heurísticas foi o mesmo nota-se que o consumo de tempo, ou seja, o tempo necessário para resolução do 15-puzzle com o algoritmo A*, está ligado intrinsecamente ao resultado informado pela heurística, uma vez que este resultado guiará o caminho a ser tomado na árvore de busca do algoritmo.

O consumo de memória torna-se uma boa métrica uma vez que, a quantidade de níveis gerados na árvore depende do valor retornado pela heurística, desta forma, quanto melhor a heurística menor será a quantidade de níveis gerado e menor será a quantidade de memória necessária para armazenar tal conjunto.

\section{Resultados e discussões}\label{sec:resultados}

Esta sessão será destinada a apresentar os resultados obtidos na execução dos casos de teste para cada heurísticas e anualizá-las utilizando as métricas na sub sessão \ref{subsec:testmetricas}.

Se a heurística $h$ satisfizer a condição adicional $h(x) \leq d (x, y) + h(y)$ para cada aresta $(x, y)$ do grafo (onde $d$ denota o comprimento dessa aresta), então $h$ é chamada monótona ou consistente. Com uma heurística consistente, A* garante encontrar um caminho ideal sem processar qualquer nó mais de uma vez e executar A* torna-se equivalente a executar o algoritmo de Dijkstra com o custo reduzido $d'(x, y) = d(x, y) + h(y) - h(x)$. Contanto que o a heurística seja consistente temos a garantia de um menor consumo de memória, dado que cada sucessor é processado uma única vez \footnote{A* search algorithm: \url{https://en.wikipedia.org/wiki/A*_search_algorithm}}. 

\subsection{Consumo de memória}

A implementação do algoritmos A* utilizada priorizou o consumo de tempo ao invés do consumo de memória, utilizando de estrutura auxiliares que possuíam cópias dos conjuntos $A$ e $F$, desta forma um grande consumo de memória já era esperado. Os resultado com "$-$" apresentados na Tabela~\ref{tab:tabmem} indicam que a quantidade de memória disponibilizada, cerca de 120 Gb, não foi suficiente para suprir a necessidade do algoritmo e fornecer o resultado.

\begin{table}[H]
\centering
\begin{tabular}{cccccc}
\hline
\textbf{Teste} & $h_{2}'(n)$ & $h_{2}'(n)$ & $h_{3}'(n)$ & $h_{4}'(n)$ & $h_{5}'(n)$ \\ \hline
1 & 23,27 Mb & 315 Mb & 9,5 Mb & 12,1 Mb & 9,6 Mb \\ \hline
2 & 12,148 Mb & 113 Mb & 9,5 Mb & 10,7 Mb & 9,6 Mb \\ \hline
3 & 241 Mb & 1,3 Gb & 9,6 Mb & 17,892 Mb & 9,6 Mb \\ \hline
4 & $-$ & $-$ & 89,1 Mb & 7,098 Gb & 124,8 Mb \\ \hline
5 & $-$ & $-$ & 1,085 Gb & 97,1 Gb & 1,69 Gb \\ \hline
6 & $-$ & $-$ & 2,669 Gb & $-$ & 8 Gb \\ \hline
7 & 4,922 Gb & 8,796 Gb & 30,3 Mb & 367 Mb & 76 Mb \\ \hline
8 & 593 Mb & 8,231 Gb & 10,4 Mb & 28 Mb & 11 Mb \\ \hline
9 & $-$ & $-$ & 125,8 Mb & 23,7 Gb & 194 Mb \\ \hline
10 & $-$ & $-$ & $-$ & $-$ & $-$ \\ \hline
\end{tabular}
\caption{Consumo de memória por teste e heurística}
\label{tab:tabmem}
\end{table}

Como é possível observar na tabela anteriormente citada, as heurísticas $h_{3}'(n)$ e $h_{5}'(n)$ foram as que produziram os melhores resultados em todos os casos de teste, sendo ainda a $h_{3}'(n)$ a melhor entre as duas. Isto é possível devido ao fato das heurísticas citadas serem consistentes, o que reduz o consumo de memória, uma vez que não serão gerados sucessores repetidos.

\subsection{Consumo de tempo}

Os casos simulados utilizam de diferentes níveis de recursos computacionais, como pode ser observado na subseção anterior, e por esse motivo eles afetam diretamente o comportamento das heurísticas. Portanto, é possível que um caso seja computacionalmente viável utilizando de uma das heurísticas, e computacionalmente inviável utilizando outra heurística. Utilizando da Tabela \ref{tab:tabtemp} como referência, podemos verificar o consumo de tempo aferido pelas cinco heurísticas utilizadas, em todos os casos testes, assim como verificar a eficiência de cada heurística. 

\begin{table}[H]
    \centering
    \begin{tabular}{cccccc}
        \hline
        \textbf{Teste} & $h_{1}'(n)$ & $h_{2}'(n)$ & $h_{3}'(n)$ & $h_{4}'(n)$ & $h_{5}'(n)$\\ \hline
        1 & 0:01.09 & 0:21.44 & 0:00.05 & 0:00.24 & 0:00.06 \\ \hline
        2 & 0:00.52 & 0:06.95 & 0:00.02 & 0:00.11 & 0:00.03 \\ \hline
        3 & 0:15.36 & 1:30.23 & 0:00.02 & 0:00.65 & 0:00.04 \\ \hline
        4 & $-$ & $-$ & 0:05.67 & 9:47.30 & 0:09.98 \\ \hline
        5 & $-$ & $-$ & 1:17.86 & 4:21:43 & 2:26.80 \\ \hline
        6 & $-$ & $-$ & 3:14.37 & $-$ & 12:10.71 \\ \hline
        7 & 5:44.46 & 10:42.72 & 0:01.43 & 0:28.89 & 0:05.88 \\ \hline
        8 & 0:40.68 & 9:46.86 & 0:00.08 & 0:02.14 & 0:00.16 \\ \hline
        9 & $-$ & $-$ & 0:08.25 & 34:58.53 & 0:16.40 \\ \hline
        10 & $-$ & $-$ & $-$ & $-$ & $-$ \\ \hline
    \end{tabular}
    \caption{Consumo de tempo, h:mm:ss ou mm:ss}
    \label{tab:tabtemp}
\end{table}

É possível observar melhores comportamentos nas heurísticas $h_{3}'(n)$ (Somatório da Distância Manhattan) e $h_{5}'(n)$ (Valor máximo entre heurísticas) em comparação com as heurísticas restantes. Além disso, notamos que as heurísticas $h_{1}'(n)$(Número de peças foras de seu lugar) e $h_{2}'(n)$(Número de peças fora de ordem na sequência numérica) conseguiram solucionar somente 50\% dos casos de teste, e esse comportamento se dá, como visto na subseção anterior, pelo fato de a memória disponibilizada não foi suficiente para suprir a necessidade do algoritmo.

\section{Conclusão}

O algoritmo A* é um algoritmo de busca de caminhos, que utilizando uma função heurística, consegue se guiar na exploração do grafo. Porém, se utilizarmos de heurísticas que não sejam consistentes, o algoritmo passa a gerar árvores muito grandes, tornando o consumo de recursos grande demais e a execução do algoritmo inviável, como demonstrado em alguns casos na seção \ref{sec:resultados}. Sendo assim, a busca pela eficiência no A* deve se pautar na utilização de heurísticas que possua estratégias para auxiliar na economia de tempo e memória.

Com base nos experimentos realizados, tivemos que a heurística $h_{3}'(n)$, Distância de Manhattan, foi a heurística que obteve melhores resultados em todos os testes, pois não só teve os menores tempos, mas também foi a que consumiu menor quantidade de memória nesse conjunto de casos testes. Notamos portanto que o uso de uma heurística consistente, como é o caso da $h_{3}'(n)$, melhora a eficiência do algoritmo A*.

Para trabalhos futuros, podemos considerar um conjunto de testes mais amplo, onde podemos analisar melhor a heurística $h_{5}'(n)$, que também se trata de uma heurística consistente, que deveria dar resultados similares ou superiores que a $h_{3}'(n)$, já que utiliza o valor dominante entre as heurísticas $h_{1}'(n)$, $h_{2}'(n)$ e $h_{3}'(n)$. Além de um conjunto de testes maior e a utilização de um banco de dados de padrões, como demonstrado por \cite{searchingpatterndatabases}, um meio efetivo para diminuir significativamente a árvore de busca gerada pelo A*.

\bibliographystyle{sbc}

\bibliography{sbc-template}

\end{document}
